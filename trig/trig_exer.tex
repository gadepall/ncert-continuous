\renewcommand{\theequation}{\theenumi}
\begin{enumerate}[label=\arabic*.,ref=\thesubsection.\theenumi]
\numberwithin{equation}{enumi}
%
%\documentclass[journal,12pt,twocolumn]{IEEEtran}
%\usepackage{setspace}
%\usepackage{gensymb}
%\usepackage{caption}
%%\usepackage{multirow}
%%\usepackage{multicolumn}
%%\usepackage{subcaption}
%%\doublespacing
%\singlespacing
%\usepackage{csvsimple}
%\usepackage{amsmath}
%\usepackage{multicol}
%%\usepackage{enumerate}
%\usepackage{amssymb}
%%\usepackage{graphicx}
%\usepackage{newfloat}
%%\usepackage{syntax}
%\usepackage{listings}
%%\usepackage{iithtlc}
%\usepackage{color}
%\usepackage{tikz}
%\usetikzlibrary{shapes,arrows}
%
%
%
%%\usepackage{graphicx}
%%\usepackage{amssymb}
%%\usepackage{relsize}
%%\usepackage[cmex10]{amsmath}
%%\usepackage{mathtools}
%%\usepackage{amsthm}
%%\interdisplaylinepenalty=2500
%%\savesymbol{iint}
%%\usepackage{txfonts}
%%\restoresymbol{TXF}{iint}
%%\usepackage{wasysym}
%\usepackage{amsthm}
%\usepackage{mathrsfs}
%\usepackage{txfonts}
%\usepackage{stfloats}
%\usepackage{cite}
%\usepackage{cases}
%\usepackage{mathtools}
%\usepackage{caption}
%\usepackage{enumerate}
%\usepackage{tfrupee}	
%\usepackage{enumitem}
%\usepackage{amsmath}
%%\usepackage{xtab}
%\usepackage{longtable}
%\usepackage{multirow}
%%\usepackage{algorithm}
%%\usepackage{algpseudocode}
%\usepackage{enumitem}
%\usepackage{mathtools}
%\usepackage{hyperref}
%%\usepackage[framemethod=tikz]{mdframed}
%\usepackage{listings}
%    %\usepackage[latin1]{inputenc}                                 %%
%    \usepackage{color}                                            %%
%    \usepackage{array}                                            %%
%    \usepackage{longtable}                                        %%
%    \usepackage{calc}                                             %%
%    \usepackage{multirow}                                         %%
%    \usepackage{hhline}                                           %%
%    \usepackage{ifthen}                                           %%
%  %optionally (for landscape tables embedded in another document): %%
%    \usepackage{lscape}     
%
%
%\usepackage{url}
%\def\UrlBreaks{\do\/\do-}
%
%
%%\usepackage{stmaryrd}
%
%
%%\usepackage{wasysym}
%%\newcounter{MYtempeqncnt}
%\DeclareMathOperator*{\Res}{Res}
%%\renewcommand{\baselinestretch}{2}
%\renewcommand\thesection{\arabic{section}}
%\renewcommand\thesubsection{\thesection.\arabic{subsection}}
%\renewcommand\thesubsubsection{\thesubsection.\arabic{subsubsection}}
%
%\renewcommand\thesectiondis{\arabic{section}}
%\renewcommand\thesubsectiondis{\thesectiondis.\arabic{subsection}}
%\renewcommand\thesubsubsectiondis{\thesubsectiondis.\arabic{subsubsection}}
%
%% correct bad hyphenation here
%\hyphenation{op-tical net-works semi-conduc-tor}
%
%%\lstset{
%%language=C,
%%frame=single, 
%%breaklines=true
%%}
%
%%\lstset{
%	%%basicstyle=\small\ttfamily\bfseries,
%	%%numberstyle=\small\ttfamily,
%	%language=Octave,
%	%backgroundcolor=\color{white},
%	%%frame=single,
%	%%keywordstyle=\bfseries,
%	%%breaklines=true,
%	%%showstringspaces=false,
%	%%xleftmargin=-10mm,
%	%%aboveskip=-1mm,
%	%%belowskip=0mm
%%}
%
%%\surroundwithmdframed[width=\columnwidth]{lstlisting}
%\def\inputGnumericTable{}                                 %%
%\lstset{
%%language=C,
%frame=single, 
%breaklines=true,
%columns=fullflexible
%}
% 
%
%\begin{document}
%%
%\tikzstyle{block} = [rectangle, draw,
%    text width=3em, text centered, minimum height=3em]
%\tikzstyle{sum} = [draw, circle, node distance=3cm]
%\tikzstyle{input} = [coordinate]
%\tikzstyle{output} = [coordinate]
%\tikzstyle{pinstyle} = [pin edge={to-,thin,black}]
%
%\theoremstyle{definition}
%\newtheorem{theorem}{Theorem}[section]
%\newtheorem{problem}{Problem}
%\newtheorem{proposition}{Proposition}[section]
%\newtheorem{lemma}{Lemma}[section]
%\newtheorem{corollary}[theorem]{Corollary}
%\newtheorem{example}{Example}[section]
%\newtheorem{definition}{Definition}[section]
%%\newtheorem{algorithm}{Algorithm}[section]
%%\newtheorem{cor}{Corollary}
%\newcommand{\BEQA}{\begin{eqnarray}}
%\newcommand{\EEQA}{\end{eqnarray}}
%\newcommand{\define}{\stackrel{\triangle}{=}}
%
%\bibliographystyle{IEEEtran}
%%\bibliographystyle{ieeetr}
%
%\providecommand{\nCr}[2]{\,^{#1}C_{#2}} % nCr
%\providecommand{\nPr}[2]{\,^{#1}P_{#2}} % nPr
%\providecommand{\mbf}{\mathbf}
%\providecommand{\pr}[1]{\ensuremath{\Pr\left(#1\right)}}
%\providecommand{\qfunc}[1]{\ensuremath{Q\left(#1\right)}}
%\providecommand{\sbrak}[1]{\ensuremath{{}\left[#1\right]}}
%\providecommand{\lsbrak}[1]{\ensuremath{{}\left[#1\right.}}
%\providecommand{\rsbrak}[1]{\ensuremath{{}\left.#1\right]}}
%\providecommand{\brak}[1]{\ensuremath{\left(#1\right)}}
%\providecommand{\lbrak}[1]{\ensuremath{\left(#1\right.}}
%\providecommand{\rbrak}[1]{\ensuremath{\left.#1\right)}}
%\providecommand{\cbrak}[1]{\ensuremath{\left\{#1\right\}}}
%\providecommand{\lcbrak}[1]{\ensuremath{\left\{#1\right.}}
%\providecommand{\rcbrak}[1]{\ensuremath{\left.#1\right\}}}
%\theoremstyle{remark}
%\newtheorem{rem}{Remark}
%\newcommand{\sgn}{\mathop{\mathrm{sgn}}}
%\providecommand{\abs}[1]{\left\vert#1\right\vert}
%\providecommand{\res}[1]{\Res\displaylimits_{#1}} 
%\providecommand{\norm}[1]{\left\Vert#1\right\Vert}
%\providecommand{\mtx}[1]{\mathbf{#1}}
%\providecommand{\mean}[1]{E\left[ #1 \right]}
%\providecommand{\fourier}{\overset{\mathcal{F}}{ \rightleftharpoons}}
%%\providecommand{\hilbert}{\overset{\mathcal{H}}{ \rightleftharpoons}}
%\providecommand{\system}{\overset{\mathcal{H}}{ \longleftrightarrow}}
%	%\newcommand{\solution}[2]{\textbf{Solution:}{#1}}
%\newcommand{\solution}{\noindent \textbf{Solution: }}
%\newcommand{\myvec}[1]{\ensuremath{\begin{pmatrix}#1\end{pmatrix}}}
%\providecommand{\dec}[2]{\ensuremath{\overset{#1}{\underset{#2}{\gtrless}}}}
%\DeclarePairedDelimiter{\ceil}{\lceil}{\rceil}
%%\numberwithin{equation}{section}
%%\numberwithin{problem}{subsection}
%%\numberwithin{definition}{subsection}
%\makeatletter
%\@addtoreset{figure}{section}
%\makeatother
%
%\let\StandardTheFigure\thefigure
%%\renewcommand{\thefigure}{\theproblem.\arabic{figure}}
%\renewcommand{\thefigure}{\thesection}
%
%
%%\numberwithin{figure}{subsection}
%
%%\numberwithin{equation}{subsection}
%%\numberwithin{equation}{section}
%%\numberwithin{equation}{problem}
%%\numberwithin{problem}{subsection}
%\numberwithin{problem}{section}
%%%\numberwithin{definition}{subsection}
%%\makeatletter
%%\@addtoreset{figure}{problem}
%%\makeatother
%\makeatletter
%\@addtoreset{table}{section}
%\makeatother
%
%\let\StandardTheFigure\thefigure
%\let\StandardTheTable\thetable
%\let\vec\mathbf
%%%\renewcommand{\thefigure}{\theproblem.\arabic{figure}}
%%\renewcommand{\thefigure}{\theproblem}
%
%%%\numberwithin{figure}{section}
%
%%%\numberwithin{figure}{subsection}
%
%
%
%\def\putbox#1#2#3{\makebox[0in][l]{\makebox[#1][l]{}\raisebox{\baselineskip}[0in][0in]{\raisebox{#2}[0in][0in]{#3}}}}
%     \def\rightbox#1{\makebox[0in][r]{#1}}
%     \def\centbox#1{\makebox[0in]{#1}}
%     \def\topbox#1{\raisebox{-\baselineskip}[0in][0in]{#1}}
%     \def\midbox#1{\raisebox{-0.5\baselineskip}[0in][0in]{#1}}
%
%\vspace{3cm}
%
%\title{ 
%%	\logo{
%Trigonometric Functions\\(Excercises)
%%	}
%}
%
%\author{ G V V Sharma$^{*}$% <-this % stops a space
%	\thanks{*The author is with the Department
%		of Electrical Engineering, Indian Institute of Technology, Hyderabad
%		502285 India e-mail:  gadepall@iith.ac.in. All content in this manual is released under GNU GPL.  Free and open source.}
%	
%}	
%
%\maketitle
%
%%\tableofcontents
%
%\bigskip
%
%\renewcommand{\thefigure}{\theenumi}
%\renewcommand{\thetable}{\theenumi}
%
%
%
%\begin{enumerate}[label=\arabic*]
%\numberwithin{equation}{enumi}
%
\item Find the radian measures corresponding to the following meausres:\\
(i) $25\degree $\\
(ii) $-47\degree 30^{'}$\\
(iii) $240\degree$\\
(iv) $520\degree$

\item Find the degree measures corresponding to the following radian measures(use $\pi$=3.14)\\
(i) $\frac{11}{16}$\\
(ii) -4\\
(iii) $\frac{5\pi}{3}$\\
(iv) $\frac{7\pi}{6}$\\

\item A wheel makes 360 revolutions in one minute. Through how many radians does it turn in one second?

\item Find the degree measure of the angle subtended at the centre of a circle of radius 100 cm by an arc of length 22 cm?

\item In a circle of diameter 40 cm, the length of a chord is 20 cm. Find the length of minor arc of the chord.

\item If in two circles, arcs of the same length subtend angles $60\degree$ and $75\degree$ at the centre, find the ratio of their radii?

\item Find the angle in radian through which a pendulum swings if its length is 75 cm and the tip describes an arc of length\\
(i) 10 cm\\
(ii) 15 cm\\
(iii) 21 cm

\item Find the values of other five trigonometric functions\\ 
1. $\cos x=-\frac{1}{2}$, x lies in third quadrant.\\
2. $\sin x= \frac{3}{5}$, x lies in second quadrant.\\
3. $\cot x= \frac{3}{4}$, x lies in third quadrant.\\
4. $\sec x= \frac{13}{5}$, x lies in fourth quadrant.\\
5. $\tan x=-\frac{5}{12}$, x lies in second quadrant.

\item Find the values of the trigonometric functions\\
1. $\sin765^{o}$\\
2. $cosec(-1410^{o})$\\
3. $\tan\frac{19\pi}{3}$\\
4. $\sin\frac{-11\pi}{3}$\\
5. $\cot\frac{-15\pi}{4}$

\item Prove that
\\1. $\sin^{2}\frac{\pi}{6}+\cos^{2}\frac{\pi}{3}-\tan^{2}\frac{\pi}{4}=-\frac{1}{2}$\\
\\2. $2\sin^{2}\frac{\pi}{6}+cosec^{2}\frac{7\pi}{6}\cos^{2}\frac{\pi}{3}=-\frac{3}{2}$\\
\\3. $\cot^{2}\frac{\pi}{6}+cosec^{2}\frac{5\pi}{6}+3\tan^{2}\frac{\pi}{6}$=6\\
\\4. $2\sin^{2}\frac{3\pi}{4}+2\cos^{2}\frac{\pi}{4}+2\sec^{2}\frac{\pi}{3}$=10\\

\item Find the value of\\
(i) $\sin75^{o}$\\
(ii) $\tan15^{o}$\\

\item Prove that \\
 $\cos(\frac{\pi}{4}-x)\cos(\frac{\pi}{4}-y)-\sin(\frac{\pi}{4}-x)\sin(\frac{\pi}{4}-y)=\sin(x+y)$\\

\item Prove that \\
\\$\frac{\tan(\frac{\pi}{4}+x)}{\tan(\frac{\pi}{4}-x)}=(\frac{1+\tan x}{1-\tan x})^{2}$\\

\item Prove that\\
\\$\frac{\cos(\pi+x)\cos(-x)}{\sin(\pi-x)\cos(\frac{\pi}{2}+x)}=\cot^{2}x$\\

\item Prove that\\
\\$\cos(\frac{3\pi}{2}+x)\cos(2\pi+x)[\cot(\frac{3\pi}{2}-x)+\cot(2\pi +x)]=1$

\item Prove that\\
\\$\sin(n+1)x\sin(n+2)x+\cos(n+1)x\cos(n+2)x=\cos x$\\

\item Prove that\\
\\$\cos(\frac{3\pi}{4}+x)-\cos(\frac{3\pi}{4}-x)=-\sqrt 2\sin x $\\

\item Prove that\\
\\$\sin^{2}6x-\sin^{2}4x=\sin2x\sin10x$\\

\item Prove that\\
\\$\cos^{2}2x-\cos^{2}6x=\sin4x\sin8x$\\

\item Prove that\\
\\$\sin2x+2\sin4x+\sin6x=4\cos^{2}x\sin4x$\\

\item Prove that\\
\\$\cot4x(\sin5x+\sin3x)= \cot x(\sin5x-\sin3x)$\\

\item Prove that\\
\\$\frac{\cos9x-\cos5x}{\sin17x-\sin3x}=-\frac{\sin2x}{\cos10x}$\\

\item Prove that\\
\\$\frac{\sin5x+\sin3x}{\cos5x+\cos3x}=\tan4x$\\

\item Prove that\\
\\$\frac{\sin x+\sin y}{\cos x+\cos y}=\tan(\frac{x-y}{2})$\\

\item Prove that\\
\\$\frac{\sin x+\sin3x}{\cos x+\cos3x}=\tan2x$\\

\item Prove that\\
\\$\frac{\sin x-\sin3x}{\sin^{2}x-\cos^{2}x}=2\sin x$\\

\item Prove that\\
\\$\frac{\cos4x+\cos3x+\cos2x}{\sin4x+\sin3x+\sin2x}=\cot3x$\\

\item Prove that\\
\\$\cot x\cot2x-\cot2x\cot3x-\cot3x\cot x=1$\\\\

\item Prove that\\
\\$\tan4x=\frac{4\tan x(1-\tan^{2}x)}{1-6\tan^{2}x+\tan^{4}x}$\\

\item Prove that\\
\\$\cos4x=1-8\sin^{2}x\cos^{2}x$\\

\item Prove that\\
\\$\cos6x=32\cos^{6}x-48\cos^{4}x+18\cos^{2}x-1$\\

\item Find the principle and general solutions of the following equations:\\
1. $\tan x=\sqrt 3$\\
2. $\sec x=2$\\
3. $\cot x=-\sqrt 3$\\
4. $cosec x=-2$\\

\item Find the general solution for each of the following equations:\\
1. $\cos4x=\cos2x$\\
2. $\cos3x+\cos x-\cos2x=0$\\
3. $\sin2x+\cos x=0$\\
4. $\sec^{2}2x=1-\tan2x$\\
5. $\sin x+\sin3x+\sin5x=0$\\

\item Prove that\\
1. 2$\cos\frac{\pi}{13}\cos\frac{9\pi}{13}+\cos\frac{3\pi}{13}+\cos\frac{5\pi}{13}=0$\\
2. $(\sin3x+\sin x)\sin x+(\cos3x-\cos x)\cos x=0$\\
3. $(\cos x+\cos y)^{2}+(\sin x-\sin y)^{2}=4\cos^{2}(\frac{x+y}{2})$\\
4. $(\cos x-\cos y)^{2}+(\sin x-\sin y)^{2}=4\sin^{2}(\frac{x-y}{2})$\\
5. $\sin x+\sin3x+\sin5x+\sin7x=4\cos x\cos2x\sin4x$\\
6. $\frac{(\sin7x+\sin5x)+(\sin9x+\sin3x)}{(\cos7x+\cos5x)+(\cos9x+\cos3x)}=\tan6x$\\
7. $\sin3x+\sin2x-\sin x=4\sin x\cos\frac{x}{2\cos\frac{3x}{2}}$\\

\item Find $\sin\frac{x}{2},\cos\frac{x}{2}$ and $\tan\frac{x}{2}$ in each of the following:\\
1. $\tan x=-\frac{4}{3}$, x in second quadrant. \\
2. $\sin x=\frac{1}{4}$, x in second quadrant.\\
3. $\cos x=-\frac{1}{3}$, x in third quadrant.\\
\end{enumerate}
%\end{document}
    
